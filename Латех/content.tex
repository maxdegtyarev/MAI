\begin{document}

\pagenumbering{gobble}
\newpage
\pagenumbering{arabic}
\setcounter{secnumdepth}{10}

(Следовательно, системы со смешанным целочисленным основанием обладают этим свойством. Наиболее общими системами такого типа являются системы со смешанным основанием, у которых $\beta_{1} = (c_{0} + 1)\beta_{0}, \beta_{2} = (c_{1} + 1)(c_{0} + 1)\beta_{0}, ..., \beta_{-1} = \beta_{0}/(c_{-1} + 1), ...)$

\section{\textit{[М21]}}
Покажите, что любое ненулевое число имеет единственное ''знакопеременное двоичное представление''
\begin{center}
$2^{e_{0}} - 2 ^{e_{1}} + ... + (-1)^{t}2^{e_{t}}$,
\end{center}
где $e_{0} < e_{1} < ... < e_{t}$

\section{\textit{[М24]}}
Покажите, что любое неотрицательное комплексное число вида а + bi, где а и b — целые числа, обладает единственным ''периодическим двоичным представлением''
\begin{center}
$1+(i)^{e_{0}} + i(1 + (i))^{e_{0}} - (1 + i)^{e_{3}} + ... + i^{-t}(1+i)^{e_{t}}$,
\end{center}
где $e_{0} < e_{1} < ... < e_{t}$ (ср. с упр. 27).
\section{\textit{[М35]}}
(Н. Г. де Брейн (N. G. de Bruijn).) Пусть $S_{0}, S_{1}, S_{2}, ...$ - множества неотрицательных целых чисел; говорят, что совокупность $\lbrace S_{0}, S_{1}, S_{2}, ...\rbrace$ обладает свойством В, если любое неотрицательное целое число n может быть единственным способом записано в виде
\begin{center}
$n = s_{0} + s_{1} + ...,	s_{j} \in S_{j}$
\end{center}
(Свойство В означает, что $0 \in S_{j}$ для всех j, поскольку n = 0 может быть представлено только как 0 + 0 + 0 + ... .) Любая система счисления со смешанным основанием $b_{0}, b_{1}, b_{2},...$ дает пример совокупности множеств, удовлетворяющих свойству В, если положить $S_{j} = \lbrace 0, B_{j},..., (b_{j} — 1)B_{j} \rbrace, где B_{j} = b_{0}b_{1} ...b{j-i}.$ В таком случае представление $n = s_{0} + s_{1} + s_{2} + ... $ очевидным образом соответствует представлению (9) этого числа по смешанному основанию. Далее, если совокупность $\lbrace S_{0}, S_{1}, S_{2},...\rbrace А_{0}, А_{1}, А_{2}, ...$ обладает
свойством В, то, каково бы ни было разбиение $А_{0}, А_{1}, А_{2}, ...$ неотрицательных целых чисел (т. е. $А_{0} \cup А_{1} \cup А_{2} U ... = \lbrace 0,1, 2,...\rbrace и A_{i} \cap A_{j} = \times\emptyset при$
% i  j, некоторые из множеств Aj могут быть пустыми), этим свойством обладает и полученная из нее путем “стягивания” совокупность {То, Т\, Т2,...}, где множество Tj состоит из всех сумм вида YlieA Si> взятых по всевозможным выборкам элементов Si 6 Si.
%Докажите, что любая последовательность {То, Т\, Т2, ...}, удовлетворяющая свой-ству В, может быть получена посредством “стягивания” некоторой совокупности {So, Si, S2, ...}, соответствующей системе счисления по смешанному основанию.

\end{document}